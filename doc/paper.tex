\documentclass{article}

\usepackage{amssymb}
\usepackage{amsmath}
\usepackage{hyperref}
\usepackage[mathscr]{eucal}
\usepackage{authblk}

\title{Sizing the bets in a focused portfolio}
%\title{Capital allocation for a focused portfolio based on fundamental analysis}

\author[$\dagger$]{Vuko Vuk\v{c}evi\'{c}}

\affil[$\dagger$]{In silico Ltd., Zagreb, Croatia, vuko.vukcevic@insilico.hr}
%\affil[$\star$]{SimScale GmbH, Munich, Germany, vvukcevic@simscale.com}

\begin{document}

\maketitle

\section{Summary}
\label{sec:summary}

This paper consist of two parts, one without the mathematics and one with the
mathematics. For a non--mathematical reader, it is best to skip the two sections
related to mathematics and numerics, namely~\autoref{sec:mathematics}
and~\autoref{sec:numerics}. The reader who is interested in mathematics should
still read the preceding sections since they provide context within which the
mathematics is built on.

\section{Motivation}
\label{sec:motivation}

Before going on to explain the motivation about this work, a bit of context
might come useful. In silico is an employee--owned engineering consultancy
company that invests its excess cash into publicly traded stocks. The excess
cash comes sporadically, and In silico has been a net buyer over the last four
years: A trend that will likely continue. When the excess cash comes in, the
following question arises: How much money to put in which stocks? That question
had been answered by looking at our fundamental analysis for each company, and
performing some vague hand calculations. That of course works well because of
all the other uncertainties in the investment process, but the motivation was
to:
\begin{enumerate}
    \item Improve the odds of us behaving rationally (i.e. minimize
    psychological misjudgments such as anchoring bias~\cite{kahneman},
    consistency and commitment tendency~\cite{cialdini}, and
    others with their combined effects~\cite{almanack}),
    \item Save time (i.e. avoid doing hand calculations).
\end{enumerate}

Since mathematics is usually good at keeping people rational, and software is
great for saving time, a decision has been made to write a software that answers
the following question: For a set of candidate companies and their current
market capitalization, each having a set of scenarios defined by a probability
and intrinsic value estimate, how much of our capital to invest in each?

\section{Introduction}
\label{sec:introduction}

The answer to this question has been given by Kelly with his widely known Kelly
formula (sometimes called Kelly criterion)~\cite{kelly}. The Kelly's approach
starts by maximizing long--term growth of capital when one is presented with
infinite amount of opportunities to bet. This work extends that idea by
considering:
\begin{enumerate}
    \item Multiple companies (stocks) in parallel,
    \item An arbitrary number of scenarios for each company,
    \item A fundamentals--based analysis (after all, a stock is an ownership
    share of business~\cite{intelligentInvestor}.
\end{enumerate}

The generalization leads to a non--linear system of equations that when solved,
yields a fraction of capital to invest in each of the candidate companies. The
mathematical derivation is presented in \autoref{sec:mathematics}, while in the
next two subsections (\autoref{sec:assumptions} and \autoref{sec:safetyMargin}),
a discussion on assumptions in a non--mathematical way, and the margin of
safety, are presented.

\subsection{Disclaimers and Assumptions}
\label{sec:assumptions}

The underlying philosophy is that one should spend the majority of his time
analyzing investments and thinking about intrinsic values under different
scenarios that might play out, independently of outside thoughts and events.
However, calculating intrinsic value of a company is more of an art than a
science, especially for a high-quality, growing businesses within one's circle
of competence. And according to Charlie Munger, Warren Buffett, Mohnish Pabrai
and the like-minded others from whom the author got the inspiration for this
work, one should focus precisely on getting such great businesses for a fair
price. That means that one shouldn't take what this approach says at face value,
and one should probably use its guidance infrequently.

During the mathematical derivation presented in \autoref{sec:mathematics}, an
assumption is made that the number of bets is very high (tends to infinity).
This work does not try to justify this assumption in a strong mathematical
form~\footnote{It is on the author's TODO list to try and prove this.}. Here's a
soft, non-mathematical reasoning on why your author thinks this is fine: The
assumption is made in order to write the equations in terms of probabilities
instead of the number of outcomes divided by the number of bets. Therefore, as
long as the input probabilities are \textit{conservatively} estimated, the
framework should still be valid. This essentially represents the most important
margin of safety~\cite{intelligentInvestor}.

\subsection{Margin of Safety}
\label{sec:safetyMargin}

In addition to the most important margin of safety mentioned above, there are
however a couple of more margins of safety embedded in the current framework.
These are:
\begin{itemize}
    \item No shorting allowed. From a purely mathematical point of view,
    shorting would be allowed. Without detailed math, it's easy to see how a
    company with a negative expected return would result in a short position.
    However, due to the asymmetry of the potential losses compared to gains,
    coupled with the usual time--frame limit that comes with shorting, the
    author decided to avoid it.
    \item No use of leverage allowed. Again, from a purely mathematical point of
    view, use of leverage would sometimes be useful. The thinking in avoiding to
    use leverage is that no--one should be in a hurry to get rich, and should
    avoid risking good night's sleep based on short--term market fluctuations,
    which are fairly hard to predict consistently (unless you work at
    Renaissance Technologies, in which case you can scratch this).
    \item No company without at least one downside scenario is allowed. By
    disallowing inputs without downside, the framework forces you to focus and
    think about what can go wrong, as opposed to dreaming about what can go
    right. However, if one is absolutely (100\%) sure that a company does not
    have a downside, then the solution is to lever up as much as possible, and
    put all the money in that one company. It is our feeling that the best way
    to handle such cases is outside of this framework. Alternatively, one can
    always model an unknown downside scenario with a small probability (say
    5--20\%) and intrinsic value of zero.
\end{itemize}

These assumptions and margins of safety are embedded into the framework in order
to try and tie the rational mathematics with common--sense from other
disciplines.

\section{Mathematics}
\label{sec:mathematics}

The following derivation mostly follows the first part of the work by Byrnes and
Barnett~\cite{byrnesBarnett}. The problem statement is repeated here for
convenience: Given a set of candidate companies, each having a set of scenarios
described by the probability $p$ and estimate of the intrinsic value
$\mathcal{V}$, calculate the optimal allocation fraction $f$ for each candidate
company by maximizing the long--term growth rate of assets. After a single
outcome (realization), the change in value of assets can be written as follows:
\begin{equation}
\label{eq:1}
    \mathcal{A}_{after}
  = 
    \mathcal{A}_{before}
    \left( 1 + \sum_{j}^{N_c} f_j k_j \right)
\end{equation} 

\noindent where $\mathcal{A}$ is the value of assets (capital), $N_c$ is the
number of candidate companies to consider, $f_j$ is the allocated fraction to
$j$th company, and $k_j$ is one of the returns for a company $j$ defined as the
relative difference between the estimated intrinsic value under a given scenario
($\mathcal{V}$) and the market capitalization at the time of investing
($\mathcal{M}$):
\begin{equation}
\label{eq:2}
    k_j = \frac{\mathcal{V}_j - \mathcal{M}_j}{\mathcal{M}_j}
\end{equation}

\noindent If a significant number of (re)allocations ($N_a$) is performed in
succession, the equation~\eqref{eq:1} can be written as follows:
\begin{equation}
\label{eq:3}
    \mathcal{A}_{N_a}
  = 
    \mathcal{A}_{0} \prod_{i_1, i_2, \hdots, i_{N_o}}
    \left( 1 + \sum_{j}^{N_c} f_j k_{ij} \right)^{n_i}
\end{equation} 

\noindent where $n_i \in {n_{i1}, n_{i2}, \hdots, n_{N_o}} $ is the number of
the $i$th outcome that occurred $n_i$ times. Note that $k_{ij}$ represents the
return of the $j$th company in the $i$th outcome. Following original Kelly's
approach, a logarithmic growth function $\mathcal{G}$ is introduced:
\begin{equation}
\label{eq:4}
    \mathcal{G} = \lim_{N_a \to \infty} \frac{1}{N_a} \ln
        \frac{\mathcal{A}_{N_a}}{\mathcal{A}_0}
\end{equation}

\noindent and the goal is to find its maximum with respect to allocation
fractions $f_j$:
\begin{equation}
\label{eq:5}
    \frac{\partial \mathcal{G}}{\partial f_j} = 0
\end{equation}

\noindent Substituting equation~\eqref{eq:1} into equation~\eqref{eq:5} results
in the following equation, after some calculus and algebra:
\begin{equation}
\label{eq:6}
    \lim_{N_a \to \infty} \frac{1}{N_a}
    \sum_{i}^{N_o} \frac{n_i k_{ij}}{1 + \sum_{j}^{N_c} f_j k_{ij}} = 0
\end{equation}

\noindent If one assumes an infinite number of (re)allocations $N_a$,
the following relation holds\footnote{This is an assumption that your author is
uncomfortable with, but \textit{feels} it is fine because of the margins of
safety embedded into the thinking that goes into assessing each investment
opportunity.}:
\begin{equation}
\label{eq:7}
    \lim_{N_a \to \infty} \frac{n_i}{N} = p_i
\end{equation}

\noindent Where $p_i$ is the probability of the $i$th outcome. For example, if
there are two companies, each with two 50--50 scenarios, there will be four
outcomes in total with the probability of each outcome equal to 25\%. Finally,
substituting equation~\eqref{eq:7} into~\eqref{eq:6} results in a system of
equations written in terms of probabilities $p_i$, expected returns for each
company in each of the outcomes $k_{ij}$, and allocation fractions for each
company $f_j$:
\begin{equation}
\label{eq:8}
    \sum_{i}^{N_o} \frac{p_i k_{ij}}{1 + \sum_{j}^{N_c} f_j k_{ij}} = 0
\end{equation}

\noindent The equation~\eqref{eq:8} represents a non--linear system of equations
in the unknown fractions $f_j$, which when solved, should yield optimal
allocation strategy for maximizing long--term growth of capital.

\section{Numerics}
\label{sec:numerics}

\noindent The equation~\eqref{eq:8} can be written succinctly as:
\begin{equation}
\label{eq:9}
    \mathcal{F}_i(f_j) = 0
\end{equation}

\noindent Because $\mathcal{F}_i$ is a non--linear equation in $f_j$,
the Newton--Raphson method is used to find a numerical solution. The method is
iterative and starts by linearizing the equation around the previous solution
from the previous iteration:
\begin{equation}
\label{eq:10}
    \mathcal{F}_i^o + \sum_{j}^{N_c} \mathcal{J}_{ij}^o (f_j^n - f_j^o) = 0
\end{equation}

\noindent where $\mathcal{J}_{ij}$ is the Jacobian of $\mathcal{F_i}$, and
superscripts $^n$ and $^o$ denote the new value and the old value, respectively.
The Jacobian is:
\begin{equation}
\label{eq:11}
    \mathcal{J}_{ij}
  = 
    - \sum_{o}^{N_o}
      \frac{p_o k_{oi} k_{oj}}{\left(1 + \sum_{j}^{N_c} f_j k_{ij} \right)^2}
\end{equation}

\noindent where subscript for outcome $i$ has been changed to $o$ in order to be
able to write the Jacobian in the standard $ij$ notation.

The numerical procedure starts by assuming the uniform allocation across
all companies, i.e. $f_j = f = \frac{1}{N_c}$. Based on the $f_j^o$ in the
current iteration, the linear system in equation~\eqref{eq:10} is solved to find
the new solution $f_j^n$. The process is repeated until sufficient level of
accuracy is reached~\footnote{Usually two significant digits is enough
considering the uncertainty and the judgment required for the inputs.}.

A practical approach is used to prevent shorting and use of leverage. Shorting
would occur if any of the fractions is negative, and is prevented by filtering
out the candidate companies with negative expected return before solving the
equations. The use of leverage is avoided by normalizing the fractions after
obtaining a numerical solution, if the sum of all fractions exceeds 1. The
author is well--aware that this is somewhat a brute--force approach, and it will
be a topic of future work to avoid shorting and use of leverage by using
inequality constraints.

\section{Example}
\label{sec:example}

\noindent As an example, consider seven candidate companies, each with two to
four scenarios. Each scenario is represented by an intrinsic value and the
probability of the scenario happening (or intrinsic value being reached at some
point in the future). Note that how these numbers are obtained is outside of the
scope of this work, although it is important to stress that the validity and
conservative assumptions behind these numbers are probably the most important
part of an investor's job. The example inputs are presented
in~\autoref{tab:companyA} to~\autoref{tab:companyF}.

\begin{table}[!ht]
\caption{Company A with current market cap of 214B USD.}
\vspace{0.25cm}
\centering
\begin{tabular}{l|c|c}
Scenario & Intrinsic value & Probability \\
\hline
Total loss & 0 USD & 5\% \\
Bear thesis & 170B USD & 30\% \\
Base thesis & 270B USD & 50\% \\
Bull thesis & 360B USD & 15\% \\
\end{tabular}%
\label{tab:companyA}%
\end{table}%

\begin{table}[!ht]
\caption{Company B with current market cap of 306M USD.}
\vspace{0.25cm}
\centering
\begin{tabular}{l|c|c}
Scenario & Intrinsic value & Probability \\
\hline
Total loss& 0 USD & 5\% \\
Bear thesis & 300M USD & 50\% \\
Base thesis & 900M USD & 45\% \\
\end{tabular}%
\label{tab:companyB}%
\end{table}%

\begin{table}[!ht]
\caption{Company C with current market cap of 34M GBP.}
\vspace{0.25cm}
\centering
\begin{tabular}{l|c|c}
Scenario & Intrinsic value & Probability \\
\hline
Total loss & 0 GBP & 10\% \\
Bear thesis & 33.5M GBP & 50\% \\
Base thesis & 135M GBP & 40\% \\
\end{tabular}%
\label{tab:companyC}%
\end{table}%

\begin{table}[!ht]
\caption{Company D with current market cap of 806M SGD.}
\vspace{0.25cm}
\centering
\begin{tabular}{l|c|c}
Scenario & Intrinsic value & Probability \\
\hline
Bear thesis & 330M SGD & 40\% \\
Base thesis & 1B SGD & 50\% \\
Bull thesis & 4B SGD & 10\% \\
\end{tabular}%
\label{tab:companyD}%
\end{table}%

\begin{table}[!ht]
\caption{Company E with current market cap of 17.6 CAD.}
\vspace{0.25cm}
\centering
\begin{tabular}{l|c|c}
Scenario & Intrinsic value & Probability \\
\hline
Total loss & 0 CAD & 5\% \\
Bear thesis & 10M CAD & 25\% \\
Base thesis & 25M CAD & 70\% \\
\end{tabular}%
\label{tab:companyE}%
\end{table}%

\begin{table}[!ht]
\caption{Company F with current market cap of 581M USD.}
\vspace{0.25cm}
\centering
\begin{tabular}{l|c|c}
Scenario & Intrinsic value & Probability \\
\hline
Total loss & 0 USD & 20\% \\
Base thesis & 1B CAD & 80\% \\
\end{tabular}%
\label{tab:companyF}%
\end{table}%


\noindent Based on the inputs, the portfolio allocation that maximizes the
long--term growth--rate of capital is presented in~\autoref{tab:results}.

\begin{table}
\caption{Portfolio that maximizes long--term growth--rate of capital.}
\vspace{0.25cm}
\centering
\begin{tabular}{l|c|c|c|c|c|c}
Company & A & B & C & D & E & F \\
\hline
Allocation fractions & 13\% & 27\% & 26\% & 12\% & 12\% & 10\% \\
\end{tabular}%
\label{tab:results}%
\end{table}%

With the obtained fractions, it is easy to obtain some useful
statistics on the portfolio, namely:
\begin{itemize}
    \item Expected gain of 61 cents for every dollar invested,
    \item Cumulative probability of loss of capital of 13\%,
    \item Permanent loss of 95\% of capital with probability of 0.001\%.
\end{itemize}

\noindent The last item is particularly interesting to your author. According to
Actuarial Life Tables in~\cite{lifeTables}, the probability of your currently 33
year--old author dying within the next year is approximately 0.25\%. That is two
orders of magnitude higher than the probability of the permanent loss of capital
for this portfolio. Considering that the portfolio has six stocks, that is a
very strong argument against excessive diversification, especially if:
\begin{itemize}
    \item One thinks of stocks as ownership shares of businesses, which implies
    long--term thinking and not being bothered by market fluctuations, 
    \item One embeds a margin of safety in different scenarios for different
    companies by e.g. recognizing that both unknown and known bad things may
    happen.
\end{itemize}

\noindent The observation about excessive diversification Poor Charlie's
Almanac~\cite{almanack}.

\section{Future Work}
\label{sec:futureWork}

\noindent The future work will focus on adding a constraint for the maximum
allowable risk of the permanent loss of capital. The idea is that an investor
could input the probability of the permanent loss of capital and the fraction of
capital one is willing to risk, and the allocation will change accordingly. For
example, an investor might be fine with losing 50\% of capital with the
probability of 0.1\% as a worst--case outcome, which should be an additional
input to the system.

\clearpage

\bibliographystyle{unsrt}
\bibliography{./references}

\end{document}

